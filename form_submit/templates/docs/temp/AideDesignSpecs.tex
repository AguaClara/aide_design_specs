%% Generated by Sphinx.
\def\sphinxdocclass{report}
\documentclass[letterpaper,10pt,spanish]{sphinxmanual}
\ifdefined\pdfpxdimen
   \let\sphinxpxdimen\pdfpxdimen\else\newdimen\sphinxpxdimen
\fi \sphinxpxdimen=.75bp\relax

\PassOptionsToPackage{warn}{textcomp}
\usepackage[utf8]{inputenc}
\ifdefined\DeclareUnicodeCharacter
% support both utf8 and utf8x syntaxes
\edef\sphinxdqmaybe{\ifdefined\DeclareUnicodeCharacterAsOptional\string"\fi}
  \DeclareUnicodeCharacter{\sphinxdqmaybe00A0}{\nobreakspace}
  \DeclareUnicodeCharacter{\sphinxdqmaybe2500}{\sphinxunichar{2500}}
  \DeclareUnicodeCharacter{\sphinxdqmaybe2502}{\sphinxunichar{2502}}
  \DeclareUnicodeCharacter{\sphinxdqmaybe2514}{\sphinxunichar{2514}}
  \DeclareUnicodeCharacter{\sphinxdqmaybe251C}{\sphinxunichar{251C}}
  \DeclareUnicodeCharacter{\sphinxdqmaybe2572}{\textbackslash}
\fi
\usepackage{cmap}
\usepackage[T1]{fontenc}
\usepackage{amsmath,amssymb,amstext}
\usepackage{babel}
\usepackage{times}
\usepackage[Sonny]{fncychap}
\ChNameVar{\Large\normalfont\sffamily}
\ChTitleVar{\Large\normalfont\sffamily}
\usepackage[,numfigreset=1,mathnumfig]{sphinx}

\fvset{fontsize=\small}
\usepackage{geometry}

% Include hyperref last.
\usepackage{hyperref}
% Fix anchor placement for figures with captions.
\usepackage{hypcap}% it must be loaded after hyperref.
% Set up styles of URL: it should be placed after hyperref.
\urlstyle{same}
\addto\captionsspanish{\renewcommand{\contentsname}{Introducción a la Tecnología AguaClara}}

\addto\captionsspanish{\renewcommand{\figurename}{Figura}}
\addto\captionsspanish{\renewcommand{\tablename}{Tabla}}
\addto\captionsspanish{\renewcommand{\literalblockname}{Lista}}

\addto\captionsspanish{\renewcommand{\literalblockcontinuedname}{proviene de la página anterior}}
\addto\captionsspanish{\renewcommand{\literalblockcontinuesname}{continué en la próxima página}}
\addto\captionsspanish{\renewcommand{\sphinxnonalphabeticalgroupname}{Non-alphabetical}}
\addto\captionsspanish{\renewcommand{\sphinxsymbolsname}{Símbolos}}
\addto\captionsspanish{\renewcommand{\sphinxnumbersname}{Numbers}}

\addto\extrasspanish{\def\pageautorefname{página}}

\setcounter{tocdepth}{0}


        \usepackage{cancel}
    

\title{AIDE Design Specifications}
\date{06 de noviembre de 2020}
\release{0.0.0}
\author{AguaClara Cornell}
\newcommand{\sphinxlogo}{\vbox{}}
\renewcommand{\releasename}{Versión}
\makeindex
\begin{document}

\ifdefined\shorthandoff
  \ifnum\catcode`\=\string=\active\shorthandoff{=}\fi
  \ifnum\catcode`\"=\active\shorthandoff{"}\fi
\fi

\pagestyle{empty}
\maketitle
\pagestyle{plain}
\sphinxtableofcontents
\pagestyle{normal}
\phantomsection\label{\detokenize{index::doc}}


Este documento está escrito y mantenido en \sphinxhref{https://github.com/AguaClara/aide\_design\_specs}{Github} via \sphinxhref{http://www.sphinx-doc.org/en/master/}{Sphinx}. Utiliza y se refiere a código y funciones de AguaClara en \sphinxhref{https://github.com/AguaClara/aguaclara}{AguaClara}. A continuación se enumeran las versiones de los programas que utilizamos:


\begin{savenotes}\sphinxattablestart
\centering
\sphinxcapstartof{table}
\sphinxcaption{Estas son las versiones de software utilizadas para compilar esta documentación.}\label{\detokenize{index:id2}}\label{\detokenize{index:software-versions}}
\sphinxaftercaption
\begin{tabular}[t]{|\X{10}{20}|\X{10}{20}|}
\hline
\sphinxstyletheadfamily 
Software
&\sphinxstyletheadfamily 
version
\\
\hline
Sphinx
&
1.7.5
\\
\hline
aguaclara
&
0.1.8
\\
\hline
Anaconda
&
4.5.4
\\
\hline
Python
&
3.6.5
\\
\hline
\end{tabular}
\par
\sphinxattableend\end{savenotes}


\chapter{Historia}
\label{\detokenize{Introduction/History:historia}}\label{\detokenize{Introduction/History:title-historia}}\label{\detokenize{Introduction/History::doc}}
AguaClara Cornell inició en 2005 como programa de la Facultad de Ingeniería Civil y Ambiental de la Universidad de Cornell que tenía como propósito diseñar tecnologías robustas de tratamiento de agua potable. Vinculado con la ONG Hondureña Agua Para el Pueblo (APP) como primer socio de implementación en Centroamérica, el programa AguaClara Cornell pretendía desarrollar soluciones sostenibles para proveer agua de calidad en la zona rural y periurbana de ese país.

Empezando con el proyecto piloto en el casco urbano del municipio de Ojojona en 2006 y 2007, la tecnología AguaClara ha evolucionado con cada proyecto mediante un proceso de integración de las experiencias del campo en las investigaciones y diseños de la Universidad. El resultado es un producto que está a la vez basado en las necesidades de las comunidades, en la experiencia de los implementadores, y en los experimentos e innovaciones del laboratorio de Cornell. Las plantas son sostenibles por cuanto se construyen empleando materiales disponibles al nivel nacional y el proceso de tratamiento y los sistemas de suministro de insumos químicos se diseñan aprovechando la fuerza de gravedad, de esta forma eliminando la necesidad de energía eléctrica. Además, las versiones más recientes de la tecnología producen agua que cumple con las normas más estrictas de calidad.

Hoy en día el programa tiene iniciativas en diversos contextos alrededor del mundo incluyendo en Nicaragua, Colombia, y la India. En Honduras, APP sigue diseñando y construyendo plantas, con 20 proyectos exitosos. Además de la implementación de la tecnología y la capacitación de nuevos operadores y juntas de agua, APP brinda asistencia técnica para las plantas existentes, distribuye los insumos para su operación, y ha facilitado la formación de una asociación de juntas de agua con plantas AguaClara, denominada ACACH (Asociación Comunitaria de AguaClara de Honduras). APP trabaja con la versión más actualizada de los diseños de la Universidad de Cornell, de esta manera facilitando la innovación en cada etapa del desarrollo del programa.


\chapter{Procesos de Tratamiento}
\label{\detokenize{Introduction/Treatment_Process:procesos-de-tratamiento}}\label{\detokenize{Introduction/Treatment_Process:title-procesos-de-tratamiento}}\label{\detokenize{Introduction/Treatment_Process::doc}}
Las plantas producen agua limpia y segura, tras la remoción de sedimentos y patógenos. La tecnología AguaClara emplea los procesos unitarios de coagulación, floculación, sedimentación, filtración rápida con arena, y desinfección con cloro (\hyperref[\detokenize{Introduction/Treatment_Process:figure-process}]{Figura \ref{\detokenize{Introduction/Treatment_Process:figure-process}}}).

\begin{figure}[htbp]
\centering
\capstart

\noindent\sphinxincludegraphics[width=650\sphinxpxdimen]{{process}.png}
\caption{Los procesos de tratamiento que se utilizan en la planta AguaClara.}\label{\detokenize{Introduction/Treatment_Process:id1}}\label{\detokenize{Introduction/Treatment_Process:figure-process}}\end{figure}


\section{El tanque de entrada}
\label{\detokenize{Introduction/Treatment_Process:el-tanque-de-entrada}}\label{\detokenize{Introduction/Treatment_Process:heading-el-tanque-de-entrada}}
El proceso inicia en el \sphinxstylestrong{tanque de entrada}, que sirve tanto para quitar del agua el material grueso como para medir el caudal de agua para la dosificación de los químicos. El tanque de entrada funciona como tanque de sedimentación en que las partículas gruesas se caen al fondo del tanque por gravedad. Debido al diseño especial de la salida, el nivel de agua en el tanque varía en proporción al caudal de agua en la planta. Este nivel de agua está conectado con el sistema semi-automático de dosificación de químicos, de tal forma que las dosis del coagulante y del cloro se mantienen aun cuando cambia el caudal de agua en la planta. Mediante el dosificador de químicos, en la salida del tanque de entrada se aplica un coagulante, que se une con el agua cruda en la \sphinxstylestrong{mezcla rápida}.


\chapter{Necesidades para Operar una Planta AguaClara}
\label{\detokenize{Introduction/Requirements:necesidades-para-operar-una-planta-aguaclara}}\label{\detokenize{Introduction/Requirements:title-necesidades-para-operar-una-planta-aguaclara}}\label{\detokenize{Introduction/Requirements::doc}}

\section{Personal e insumos}
\label{\detokenize{Introduction/Requirements:personal-e-insumos}}\label{\detokenize{Introduction/Requirements:heading-personal-e-insumos}}\begin{itemize}
\item {} 
Operador capacitado en el manejo de la tecnología AguaClara presente siempre cuando la planta está funcionando, las 24 horas en la mayoría de los casos.

\item {} 
Dosis adecuada y constante de un químico coagulante (sulfato de aluminio o policloruro de aluminio) para la remoción de turbiedad.

\item {} 
Dosis adecuada y constante de hipoclorito de calcio (cloro) para la desinfección.

\end{itemize}


\section{Equipo básico de laboratorio}
\label{\detokenize{Introduction/Requirements:equipo-basico-de-laboratorio}}\label{\detokenize{Introduction/Requirements:heading-equipo-basico-de-laboratorio}}\begin{itemize}
\item {} 
Un turbidímetro portátil, la herramienta más fundamental para la operación de la planta (\hyperref[\detokenize{Introduction/Requirements:figure-microtpi}]{Figura \ref{\detokenize{Introduction/Requirements:figure-microtpi}}}). Este instrumento se usa para medir la cantidad de sedimento que trae el afluente a la planta para elegir la dosis de coagulante, para evaluar el rendimiento de cada proceso con medidas de su efluente, y para registrar la calidad de agua que la planta produce durante el día. En las plantas AguaClara existentes se ha usado el MicroTPI turbidímetro portátil con luz infrarroja de HF Scientific con un rango efectivo de 0.02 UTN a 1100 UTN.

\item {} 
El kit de calibración del turbidímetro para asegurar mediciones precisas.

\item {} 
Comparador de cloro para comprobar con regularidad que la concentración de cloro libre residual está dentro del rango aceptable en la red de distribución.

\item {} 
Probetas para medir volúmenes pequeños de líquidos, especialmente para pruebas del sistema de dosificación de químicos.

\item {} 
Una escala para medir las masas de químicos en polvo para la preparación de las soluciones madres del coagulante y cloro.

\item {} 
Un cronómetro para medir caudales de agua y químicos.

\end{itemize}

\begin{figure}[htbp]
\centering
\capstart

\noindent\sphinxincludegraphics[width=200\sphinxpxdimen]{{microtpi}.png}
\caption{El MicroTPI turbidímetro de HF Scientific.}\label{\detokenize{Introduction/Requirements:id1}}\label{\detokenize{Introduction/Requirements:figure-microtpi}}\end{figure}


\chapter{La Herramienta de Diseño Automática}
\label{\detokenize{Introduction/AIDE_Tools:la-herramienta-de-diseno-automatica}}\label{\detokenize{Introduction/AIDE_Tools:title-la-herramienta-de-diseno-automatica}}\label{\detokenize{Introduction/AIDE_Tools::doc}}

\section{Concepto: diseño paramétrico generalizado}
\label{\detokenize{Introduction/AIDE_Tools:concepto-diseno-parametrico-generalizado}}\label{\detokenize{Introduction/AIDE_Tools:heading-concepto-diseno-parametrico-generalizadoa}}
\sphinxstyleemphasis{La Herramienta de Diseño de Infraestructura AguaClara} conocido por conocido por su abreviatura en inglés como AIDE esta elaborada usando Onshape. Los algoritmos del diseño hidráulico se escriben usando FeatureScript. La geometría de la planta se dibuja usando la potencia del diseño paramétrico. El programa se ejecuta para producir un modelo en tres dimensiones que contiene todas las estructuras, tuberías, y accesorios hidráulicos para una planta AguaClara de esa capacidad. De este modelo se puede sacar la información necesaria para la construcción de esa instalación, tal como los cortes que se adaptan para los planos de construcción y la configuración de todos los accesorios y tubería.


\chapter{Medidor Lineal de Caudal (LFOM)}
\label{\detokenize{Entrance_Tank/LFOM:medidor-lineal-de-caudal-lfom}}\label{\detokenize{Entrance_Tank/LFOM:title-lfom}}\label{\detokenize{Entrance_Tank/LFOM::doc}}

\section{El vertedero tipo Sutro}
\label{\detokenize{Entrance_Tank/LFOM:el-vertedero-tipo-sutro}}\label{\detokenize{Entrance_Tank/LFOM:heading-el-vertedero-tipo-sutro}}
El vertedero tipo Sutro es una apertura que, al pasar agua de un lado al otro, mantiene una relación \sphinxstylestrong{lineal entre} el nivel de agua y el caudal que está pasando (Ilustración 9). En la planta AguaClara se imita la función del vertedero tipo Sutro con un diseño de orificios que crea la misma relación en la salida del tanque de entrada. El cálculo de este diseño está basado en el principio de Torricelli, el cual dice que el caudal que pasa por cada orificio sumergido es proporcional a la raíz cuadrada de la altura de agua arriba del centro del orificio:
\begin{equation}\label{equation:Entrance_Tank/LFOM:orifice_equation}
\begin{split}  Q = A \sqrt{2gh}\end{split}
\end{equation}
\begin{DUlineblock}{0em}
\item[] Donde:
\item[] \(Q\) = el caudal que pasa por el orifico
\item[] \(A\) = el área del orificio
\item[] \(g\) = la aceleración debida a la gravedad
\item[] \(h\) = la altura del agua arriba del centro del orificio
\end{DUlineblock}

\begin{figure}[htbp]
\centering
\capstart

\noindent\sphinxincludegraphics[width=500\sphinxpxdimen]{{sutro}.png}
\caption{La forma de un vertedero tipo Sutro.}\label{\detokenize{Entrance_Tank/LFOM:id1}}\label{\detokenize{Entrance_Tank/LFOM:figure-sutro}}\end{figure}


\section{Diseño de los orificios}
\label{\detokenize{Entrance_Tank/LFOM:diseno-de-los-orificios}}\label{\detokenize{Entrance_Tank/LFOM:heading-diseno-de-los-orificios}}
Para diseñar el Medidor Lineal de Caudal (LFOM por sus siglas en inglés) el algoritmo de la herramienta de diseño optimiza el número de agujeros en cada fila de tal forma que el nivel cero (debajo de la primera fila de agujeros) corresponde un caudal de cero, el nivel máximo (la parte arriba de la última fila de orificios) corresponde al caudal máximo de diseño, y la relación entre el caudal y la altura de la superficie del agua entre los dos puntos es lineal.


\begin{savenotes}\sphinxattablestart
\centering
\sphinxcapstartof{table}
\sphinxcaption{Diseño del medidor lineal de caudal}\label{\detokenize{Entrance_Tank/LFOM:id2}}\label{\detokenize{Entrance_Tank/LFOM:table-diseno-del-medidor-lineal-de-caudal}}
\sphinxaftercaption
\begin{tabular}[t]{|\X{50}{60}|\X{10}{60}|}
\hline

Rango de niveles de agua (distancia vertical entre el nivel cero y el nivel máximo)
&
20.0 cm
\\
\hline
Diámetro de los agujeros
&
1.59 cm
\\
\hline
Separación entre las filas de agujeros (centro a centro)
&
1.67 cm
\\
\hline
Número de agujeros en cada fila, empezando con la fila inferior
&
{[}17.0, 4.0, 6.0, 3.0, 4.0, 3.0, 3.0, 3.0, 3.0, 2.0, 3.0, 1.0{]}
\\
\hline
Altura de cada fila arriba del nivel cero, empezando con la fila inferior
&
{[}“7.94 mm”, “2.47 cm”, “4.14 cm”, “5.82 cm”, “7.49 cm”, “9.16 cm”, “10.84 cm”, “12.51 cm”, “14.18 cm”, “15.86 cm”, “17.53 cm”, “19.21 cm”{]}
\\
\hline
\end{tabular}
\par
\sphinxattableend\end{savenotes}

Con el medidor de caudal lineal de 20cm, cada incremento en el caudal provoca el mismo incremento en el nivel de agua en el tanque de entrada.

\begin{figure}[htbp]
\centering
\capstart

\noindent\sphinxincludegraphics[width=500\sphinxpxdimen]{{lfom20}.png}
\caption{Funcionamiento de un medidor lineal de caudal (LFOM) de 20 cm.}\label{\detokenize{Entrance_Tank/LFOM:id3}}\label{\detokenize{Entrance_Tank/LFOM:figure-lfom20}}\end{figure}

\sphinxhref{https://github.com/AguaClara/aide\_design\_specs/releases/latest}{Las versiones de PDF y LaTeX} %
\begin{footnote}[1]\sphinxAtStartFootnote
Las versiones de PDF y LaTeX pueden contener rarezas visuales porque se genera automáticamente. El sitio web es la forma recomendada de leer este documento. \sphinxhref{https://github.com/AguaClara/aide\_design\_specs}{Por favor visite nuestro GitHub} para enviar un problema, contribuir o comentar.
%
\end{footnote}.
\subsubsection*{\sphinxstylestrong{Notas}}



\renewcommand{\indexname}{Índice}
\printindex
\end{document}